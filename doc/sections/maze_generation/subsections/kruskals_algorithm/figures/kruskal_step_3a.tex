\begin{figure}[H]
    \centering
    \begin{tikzpicture}[scale=0.8]
        \draw[step=1cm, ultra thin, black] (0,0) grid (5,5);
        \draw (1.5,1.5) circle[radius=4mm];

        \fill[pattern=north east lines, pattern color=gray] (0,1) rectangle ++(1,1);
        \fill[pattern=north east lines, pattern color=gray] (1,0) rectangle ++(1,1);
        \fill[pattern=north east lines, pattern color=gray] (1,2) rectangle ++(1,1);
        \fill[pattern=north east lines, pattern color=gray] (2,1) rectangle ++(1,1);

        \newcounter{char_kruskal_3a}
        \setcounter{char_kruskal_3a}{1}
        \foreach \x in {0,...,4} {
            \foreach \y in {0,...,4} {
                \ifnum\value{char_kruskal_3a}<26
                    \node at (\x + 0.5, 4.5 - \y) {\small\textbf{\Alph{char_kruskal_3a}}};
                    \stepcounter{char_kruskal_3a}
                \fi
            }
        }
    \end{tikzpicture}
    \caption{\centering Losowa komórka (oznaczona kółkiem) i jej sąsiedzi (zakreskowani).}
    \label{fig:kruskal_step_3_a}
\end{figure}
