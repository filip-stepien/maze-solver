\begin{figure}[H]
    \centering
    \begin{subfigure}{0.45\textwidth}
        \centering
        \begin{tikzpicture}[scale=0.8]
            \draw[step=1cm, ultra thin, black] (0,0) grid (5,5);

            \fill[gray] (1,1) rectangle ++(1,1);

            \newcounter{char_kruskal_3ca}
            \setcounter{char_kruskal_3ca}{1}
            \foreach \x in {0,...,4} {
                \foreach \y in {0,...,4} {
                    \ifnum\value{char_kruskal_3ca}<26
                        \node at (\x + 0.5, 4.5 - \y) {\small\textbf{\Alph{char_kruskal_3ca}}};
                        \stepcounter{char_kruskal_3ca}
                    \fi
                }
            }

            \fill[white] (0.1,1.1) rectangle ++(0.8, 0.8);
            \fill[white] (1.1,0.1) rectangle ++(0.8, 0.8);
            \fill[white] (1.1,2.1) rectangle ++(0.8, 0.8);
            \fill[white] (2.1,1.1) rectangle ++(0.8, 0.8);

            \fill[pattern=north east lines, pattern color=darkgray] (1,0) rectangle ++(1,1);
            \fill[pattern=north east lines, pattern color=darkgray] (2,1) rectangle ++(1,1);
            \fill[pattern=north east lines, pattern color=darkgray] (3,0) rectangle ++(1,1);

            \node at (0.5, 1.5) {\small\textbf{H}};
            \node at (1.5, 0.5) {\small\textbf{H}};
            \node at (1.5, 2.5) {\small\textbf{H}};
            \node at (2.5, 1.5) {\small\textbf{H}};

            \draw[ultra thick] (0, 1) -- (0, 2);
            \draw[ultra thick] (1, 0) -- (1, 1);
            \draw[ultra thick] (2, 0) -- (2, 1);
            \draw[ultra thick] (3, 1) -- (3, 2);
            \draw[ultra thick] (3, 1) -- (3, 2);
            \draw[ultra thick] (1, 2) -- (1, 3);
            \draw[ultra thick] (2, 2) -- (2, 3);

            \draw[ultra thick] (1, 0) -- (2, 0);
            \draw[ultra thick] (1, 3) -- (2, 3);
            \draw[ultra thick] (0, 2) -- (1, 2);
            \draw[ultra thick] (0, 1) -- (1, 1);
            \draw[ultra thick] (2, 1) -- (3, 1);
            \draw[ultra thick] (2, 2) -- (3, 2);

            \draw (2.5,0.5) circle[radius=4mm];
        \end{tikzpicture}
        \caption{\centering Wylosowanie kolejnej komórki.}
        \label{fig:kruskal_step_3_c_a}
    \end{subfigure}
    \begin{subfigure}{0.45\textwidth}
        \centering
        \begin{tikzpicture}[scale=0.8]
            \draw[step=1cm, ultra thin, black] (0,0) grid (5,5);
            \newcounter{char_kruskal_3cb}
            \setcounter{char_kruskal_3cb}{1}

            \fill[lightlightgray] (2,0) rectangle ++(1, 1);

            \fill[gray] (1,1) rectangle ++(1,1);

            \foreach \x in {0,...,4} {
                \foreach \y in {0,...,4} {
                    \ifnum\value{char_kruskal_3cb}<26
                        \node at (\x + 0.5, 4.5 - \y) {\small\textbf{\Alph{char_kruskal_3cb}}};
                        \stepcounter{char_kruskal_3cb}
                    \fi
                }
            }

            \fill[white] (0.1,1.1) rectangle ++(0.8, 0.8);
            \fill[white] (1.1,0.1) rectangle ++(0.8, 0.8);
            \fill[white] (1.1,2.1) rectangle ++(0.8, 0.8);
            \fill[white] (2.1,1.1) rectangle ++(0.8, 0.8);

            \node at (0.5, 1.5) {\small\textbf{H}};
            \node at (1.5, 0.5) {\small\textbf{H}};
            \node at (1.5, 2.5) {\small\textbf{H}};
            \node at (2.5, 1.5) {\small\textbf{H}};

            \draw[ultra thick] (0, 1) -- (0, 2);
            \draw[ultra thick] (1, 0) -- (1, 1);
            \draw[ultra thick] (2, 0) -- (2, 1);
            \draw[ultra thick] (3, 1) -- (3, 2);
            \draw[ultra thick] (3, 1) -- (3, 2);
            \draw[ultra thick] (1, 2) -- (1, 3);
            \draw[ultra thick] (2, 2) -- (2, 3);

            \draw[ultra thick] (1, 0) -- (2, 0);
            \draw[ultra thick] (1, 3) -- (2, 3);
            \draw[ultra thick] (0, 2) -- (1, 2);
            \draw[ultra thick] (0, 1) -- (1, 1);
            \draw[ultra thick] (2, 1) -- (3, 1);
            \draw[ultra thick] (2, 2) -- (3, 2);
        \end{tikzpicture}
        \caption{\centering Pozostawienie przejścia.}
        \label{fig:kruskal_step_3_c_b}
    \end{subfigure}
    \caption{\centering \centering Wylosowana komórka ma już dwóch takich samych sąsiadów — scalanie nie następuje. Przejście poglądowo oznaczono jasnoszarym kolorem, aby zaznaczyć, że zostało odwiedzone.
}
    \label{fig:kruskal_step_3_c}
\end{figure}