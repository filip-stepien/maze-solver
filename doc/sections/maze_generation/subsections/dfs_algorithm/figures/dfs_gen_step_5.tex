\begin{figure}[H]
    \centering
    \begin{tikzpicture}[scale=0.8]
        \foreach \x in {0,...,4} {
            \foreach \y in {0,...,2} {
                \fill[gray] (\x,\y) rectangle ++(1,1);
            }
        }
        \fill[white] (0, 2) rectangle ++(1,1);
        \fill[white] (1, 2) rectangle ++(1,1);
        \fill[white] (2, 2) rectangle ++(1,1);

        \fill[white] (0,0) rectangle ++(1,1);
        \fill[pattern=north east lines, pattern color=black] (0,0) rectangle ++(1,1);

        \fill[white] (2,0) rectangle ++(1,1);
        \fill[white] (2,1) rectangle ++(1,1);
        \fill[white] (3,0) rectangle ++(1,1);
        \fill[white] (4,0) rectangle ++(1,1);
        \fill[white] (4,1) rectangle ++(1,1);
        \fill[white] (4,2) rectangle ++(1,1);

        \draw[->, very thick, black, dashed] (2.5,0.6) -- (2.5,2.4);
        \draw[->, very thick, black, dashed] (2.4,0.5) -- (0.6,0.5);

        \draw[->, very thick, black, dashed] (4.4,0.5) -- (2.6,0.5);

        \draw[->, very thick, black, dashed] (0.5,2.4) -- (0.5,0.6);

        \draw[->, very thick, black, dashed] (2.65,2.62) -- (4.32,2.62);
        \draw[->, very thick, dashed] (2.4,2.5) -- (0.6,2.5);

        \draw[->, very thick, dashed] (4.5,2.6) -- (4.5,4.4);
        \draw[->, very thick, dashed] (4.6,2.5) -- (6.4,2.5);
        \draw[->, very thick, black, dashed] (4.32,2.38) -- (2.65,2.38);
        \draw[->, very thick, dashed] (4.5,2.4) -- (4.5,0.6);

        \draw[dashed] (6,2) rectangle ++(1,1);
        \draw[dashed] (4,4) rectangle ++(1,1);

        \draw (4.5,2.5) circle[radius=4mm];

        \draw[step=1cm,ultra thin,black] (0,0) grid (5,3);
    \end{tikzpicture}
    \caption{\centering Stan labiryntu po kilku następnych iteracjach. Algorytm zacznie wracać, gdyż nie istnieje żaden prawidłowy ruch generujący przejście - ruchy w górę lub prawo spowodują wygenerowanie przejścia poza planszę, a ruchy w lewo bądź dół - powstanie pętli.}
    \label{fig:dfs_gen_step_5}
\end{figure}