\begin{figure}[H]
    \centering
    \begin{tikzpicture}[scale=0.8]
        \foreach \x in {0,...,4} {
            \foreach \y in {0,...,2} {
                \fill[gray] (\x,\y) rectangle ++(1,1);
            }
        }
        \fill[white] (0, 2) rectangle ++(1,1);
        \fill[white] (1, 2) rectangle ++(1,1);
        \fill[white] (2, 2) rectangle ++(1,1);

        \fill[white] (0,0) rectangle ++(1,1);
        \fill[pattern=north east lines, pattern color=black] (0,0) rectangle ++(1,1);

        \fill[white] (2,0) rectangle ++(1,1);
        \fill[pattern=north east lines, pattern color=black] (2,0) rectangle ++(1,1);

        \draw[->, very thick, black, dashed] (0.5,2.4) -- (0.5,0.5);

        \draw[->, very thick, black] (2.5,2.5) -- (4.5,2.5);
        \draw[->, very thick, black] (2.5,2.5) -- (2.5,0.5);
        \draw[->, very thick, black] (2.5,2.5) -- (2.5,4.5);
        \draw[->, very thick, black] (2.5,2.5) -- (0.6,2.5);

        \draw[dashed] (2,4) rectangle ++(1,1);

        \draw (2.5,2.5) circle[radius=4mm];

        \draw[step=1cm,ultra thin,black] (0,0) grid (5,3);
    \end{tikzpicture}
    \caption{\centering Wybór kolejnego pola. Ruchy występujące w poprzednich krokach, ale nie brane pod uwagę w bieżącej iteracji oznaczono przerywaną linią.}
    \label{fig:dfs_gen_step_4}
\end{figure}