\documentclass{article}
\usepackage[utf8]{inputenc}
\usepackage{tikz}
\usepackage[polish]{babel}
\usepackage{caption}
\usepackage[T1]{fontenc}
\usepackage{standalone}
\title{Maze solver}
\author{Rafał Grot \and Filip Stępień}

\begin{document}
\maketitle

\begin{abstract}
  Generowanie dwuwymiarowego labiryntu oraz wyszukiwanie ścieżki w dwuwymiarowym labirynice z zadanaego pola do dowolnego innego pola.
\end{abstract}

% Parametry siatki
\newcommand{\GridWidth}{8}   % liczba kolumn
\newcommand{\GridHeight}{6}  % liczba wierszy


\section{Labirynt}

Labirynt zbudownay jest z pól które reprezentują przejście lub ścianę

\begin{figure}[ht]
  \begin{center}
    \includestandalone{figures/maze_example}
  \end{center}
  \caption{Przykładowy labirynt}
\end{figure}

\section{Generowanie labiryntu}
\subsection{manual}
\subsection{depth first search}
\subsection{krushal}
\subsection{Prim's}

\section{Szukanie drogi}
\subsection{depth first search}
\subsection{breadth first search}
\subsection{A*}


\end{document}